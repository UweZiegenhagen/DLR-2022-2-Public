\documentclass[12pt,ngerman]{scrartcl}
\usepackage[utf8]{inputenc} 
\usepackage[T1]{fontenc}
\usepackage{babel}

\usepackage{datatool}
\usepackage[]{booktabs}

\DTLsetseparator{;}
\DTLloaddb{betrag}{Beispiel.csv}

%schlecht
%\newcommand{\katze}{Melli\ }

%gut
\usepackage{xspace}
\newcommand{\katze}{Melli\xspace}


\begin{document}

\begin{table}
\caption{Punktübersicht}
\centering
\begin{tabular}{llllll}
\bfseries Name &
\bfseries Vorname &
\bfseries Strasse &
\bfseries PLZ  &
\bfseries Ort  &
\bfseries Betrag \\
\DTLforeach{betrag}{%
\name=Name,\vorname=Vorname,\strasse=Strasse,\plz=PLZ,\ort=Ort,\betrag=Betrag}{%
\name & \vorname & \strasse & \plz & \ort & \betrag \\  \DTLiflastrow{\midrule}{}
}  
\DTLforeach{betrag}{%
\name=Name,\vorname=Vorname,\strasse=Strasse,\plz=PLZ,\ort=Ort,\betrag=Betrag}{%
\name & \vorname & \strasse & \plz & \ort & \betrag \\ }
\end{tabular}
\end{table}

\DTLforeach{betrag}{%
\name=Name,\vorname=Vorname,\betrag=Betrag}{%

\vorname\ \name\ hat \betrag\ bezahlt.}

\katze wacht gern sehr früh auf.

Unsere Katze heißt \katze.


\end{document}
