%!TEX TS-program = Arara
% arara: pdflatex: {shell: yes}
\documentclass[ngerman]{beamer}

\usepackage{booktabs}
\usepackage{babel}
\usepackage{graphicx}
\usepackage{csquotes}
\usepackage{xcolor}

\begin{document}

\begin{frame}

\begin{itemize}
\item Einbindung und Anwendung der DLR Corporate Design Vorlage (.sty files) in Overleaf-Projekte
\item Anleitung zur Übertragung eines Posters in LaTeX, besonders bzgl. Transparenzen, Formatierung und flexibler Boxränder für Bilder/Text
\end{itemize}
\end{frame}


\begin{frame}[allowframebreaks]

\begin{itemize}
\item Was ist eine gute Open Source Software, um Abbildungen (wie zum Beispiel in Flussdiagramm in Powerpoint) zu erstellen und diese als eps abzuspeichern? In PP lassen sich die Abbildungen nur als Grafiken speichern und erscheinen dann nicht als Vektor-Grafik. Gibt es ggf. eine direkte Latex-Lösung, um ähnlich wie in PP solche Abbildungen zu erstellen?
 
\item Ein paar Beispiele zur Definition von eigenen Befehlen. Hilfreich wären zunächst einfache Befehle, wie z.B. das Einbinden einer Abbildung (ohne durch die entsprechende Syntax mehrere Zeilen zu füllen). Gerne aber auch komplexere Anwendungsbeispiele, um zu erkennen, was alles möglich ist.
 
\item  
Kopplung von Jabref und Latex: Wie kann ich in Latex den Zitierstil ändern? Ganz konkret die Frage: Kann ich anstelle der üblichen Platzhalter bei Querverweise im Text wie z.B. [1] auch den Bibtexkey aus Jabref verwenden? Ich verwende beispielsweise als Bibtexkey immer den ersten Autornamen + Erscheinungsjahr.
 
\item 
Ich hatte in der Vergangenheit beim Öffnen der aus Latex generierten PDF öfter mal das Problem, dass die PDF nicht auf der ersten Seite, sondern mittendrin geöffnet wird. Außerdem führten die Hyperlinks im pdf Navigator nicht zu den entsprechenden Kapiteln, sondern teilweise um eine Seite versetzt.
 
\item 
Ein weiteres Problem ist relativ ähnlich. Ich verwende den Sumatra PDF-Reader in Kombination mit TexnicCenter. In der Vergangenheit ist hier auch schon das Problem aufgetreten, dass ich beim Klicken auf eine bestimmte Zeile in der PDF nicht auf die entsprechende Zeile in der Tex-Datei weitergeleitet werde. Woran liegt das und kann ich das irgendwie einstellen?
 
\item 
Wie lässt sich mit Latex ein Deckblatt für die Diss erstellen und in das „normale“ Dokument einbinden? Wie funktioniert das? In diesem Zusammenhang habe ich noch keine Kenntnisse.
\end{itemize}
\end{frame}


\begin{frame}[allowframebreaks]

\begin{itemize}
\item eigene kleine Klassen erstellen. Ich arbeite schon viel mit benutzerdefinierten \textbackslash newcommand und würde so die Anpassbarkeit noch weiter steigern
Für meine \textbackslash  verwende ich auch oft mathematische Abhängigkeiten, um Längen/Positionen zu berechnen. Dabei ermangelt es mir aber oft an Befehlen für einfache Berechnungen wie Input-Parameter zu addieren/multiplizieren/runden. Bisher habe ich meist alles über \textbackslash  pgfmathsetmacro realisiert, wäre aber über Anregungen dankbar. Da waren nämlich einige unsaubere Sachen dabei (konnte keine Double-Zahlen verrechnen und bin dann über \textbackslash  newdimen mit der Einheit pt gegangen, wo ich nicht weiß, ob das so best practice ist).
\item 
Auch habe ich dabei oft ineinander verschachtelte if-Bedinungen verwendet. Hier würde mich einerseits interessieren, welche if-Statements ich wie abprüfe (Strings, Integer, Double, etc.) und ob es auch sowas wie switch-Statements gibt um nicht zehn if’s ineinander zu verbauen.

\item Ich würde auch gerne ein bisschen mehr die Möglichkeiten mit makeatletter/makeatother eintauchen und eigenen Änderungen z.B. bei Kapiteln, Kapitelmarken oder Seitenumbrüchen vorzunehmen.

\item Ich habe über das listings-Package auch schon mal versucht die Formatierung einer eigenen Programmiersprache zu realisieren, konnte aber nicht alle Einstellungen wie gewünscht vornehmen.

\item 
Außerdem arbeite ich gerne mit Tikz. Sehr interessant wäre hier z.B. die Erstellung eigener Fill Pattern. Ich habe mir bislang in einigen Foren-Einträgen Lösungen zusammengesucht, aber die Herangehensweise dazu nie richtig verstanden.

\item Ich nutze auch pgfplots sehr intensiv, habe dabei aber bei der Positionierung immer wieder ein paar Probleme (remember picture, overlay und \textbackslash  begin{axis}[xshift =xx cm]).
Ebenso kommt es bei der Verwendung von fillbetween immer wieder zu Fehler, wenn ich Flächen zwischen zwei Kurven füllen möchte (Dimension too large).
\item 
Ich habe auch noch kein wirklich gutes Packages zur Erstellung von Pie-Charts gefunden, arbeite bisher immer pgf-pie, aber so richtig glücklich bin ich damit nicht.

\end{itemize}
\end{frame}


\end{document}