%!TEX TS-program = Arara
% arara: pdflatex: {shell: yes}
% arara: pdflatex: {shell: yes}
\documentclass[12pt,ngerman,parskip=half]{scrreprt}

%\usepackage[utf8]{inputenc} braucht man nicht mehr, da Standard
\usepackage[T1]{fontenc}
\usepackage{booktabs} % Tolle Tabellen
\usepackage{babel} % Trennmuster, ``eindeutschen''
\usepackage{graphicx} % Bilder einbinden
\usepackage{csquotes} % \enquote{} Befehl für Gänsefüßchen
\usepackage{paralist} % Kurze Aufzählungen und Listen
\usepackage{xcolor} % Farben
\usepackage{blindtext} % Dummy-Text zur Beurteilung des Textsatzes
\usepackage{hyperref} % Links im Dokument
\usepackage{microtype} % Mikrotypografie für noch besseren Satz
\usepackage{subcaption} % Bilder nebeneinander

\setcapindent{2em} % kein Abstand bei mehrzeiligen Captions

\author{Donald Duck}
\title{Meine Dissertation}
\date{Entenhausen, den 17.05.2022}

\includeonly{Kapitel-02,Kapitel-03} % Nur Kapitel 2 und 3 neu kompilieren, aber Seitenzahlen richtig setzen

\begin{document}
\begin{titlepage}

{\Large\bfseries DLR - Deutsches Zentrum für Luft- und Raumfahrt} \vspace*{1cm}

\fbox{\includegraphics[width=3cm]{Bilder/Katze}}

\vspace*{2cm}

\begin{center}
\huge\bfseries Dissertation zum Dr. Ing.  \\ 
\enquote{Thermoelastische Brücken im Quantenbereich}
\end{center}

\vspace*{2cm}

\begin{center}
\large\bfseries Donald Duck \\ 
Matrikelnummer 123456
\end{center}

\vfill

\begin{tabular}{lp{0.35\textwidth}r}
Erstgutachter: Daniel Düsentrieb & &Berlin, den \today \\ 
Zweitgutachter: Mickey Mouse \\
\end{tabular}

\end{titlepage}


\tableofcontents

\listoffigures

\listoftables

\chapter{Bilder einfügen}

\blindtext

\blindtext


\begin{figure}[hbtp!]%\centering%
\begin{center}
\includegraphics[width=0.9\textwidth]{Bilder/Katze}
\end{center}
\caption{Meine Katze, \blindtext}
\end{figure}
% h = here, t = top, b = bottom, p = Leerseite


\blindtext

\begin{figure}
\centering
\subcaptionbox{Eine Katze \label{cat1}}
{\includegraphics[width=0.49\textwidth]{Bilder/Katze}}
\subcaptionbox{Die selbe Katze \label{cat2}}
{\includegraphics[width=0.49\textwidth]{Bilder/Katze}}
%\caption{Zwei Katzenbilder}\label{katzenbilder}
\subcaptionbox{Eine Katze \label{cat3}}
{\includegraphics[width=0.49\textwidth]{Bilder/Katze}}
\subcaptionbox{Die selbe Katze \label{cat4}}
{\includegraphics[width=0.49\textwidth]{Bilder/Katze}}
\caption{Zwei Katzenbilder, \blindtext[1]}\label{katzenbilder}
\end{figure}

Abbildung \ref{cat1} auf Seite \pageref{katzenbilder}

Abbildung \ref{cat2} auf Seite \pageref{katzenbilder}

Abbildung \ref{katzenbilder} auf Seite \pageref{katzenbilder}



% Ausgabe verschiedener Längen
\the\textwidth

\the\linewidth

\the\columnwidth

% Ausgabe verschiedener Längen, linewidth ist kürzer, da innerhalb einer Aufzählung

\begin{itemize}
	\item \the\textwidth
	\item \the\linewidth
	\item \the\columnwidth
\end{itemize}


\chapter{Bilder einfügen alternativ}

\blindtext  sdfdf dfgg sdg dfgdfgs sdfdf dfgg sdg dfgdfgs sdfdf dfgg sdg dfgdfgs sdfdf dfgg sdg dfgdfgs sdfdf dfgg sdg dfgdfgs sdfdf dfgg sdg dfgdfgs sdfdf dfgg sdg dfgdfgs sdfdf dfgg sdg dfgdfgs sdfdf dfgg sdg dfgdfgs 

%Minipage, um Bild und Unterschrift zusammen zu halten
\begin{minipage}{\textwidth}
\begin{center}
\includegraphics[width=0.9\textwidth,angle=-5]{Bilder/Katze}
\captionof{figure}{Meine Katze}
\end{center}
\end{minipage}


\blindtext


\chapter{Bilder in Tabellen}

\begin{tabular}{cc}
\includegraphics[width=0.45\textwidth]{Bilder/Katze} & \includegraphics[height=0.4\textwidth]{Bilder/Katze} \\
\includegraphics[width=0.49\textwidth]{Bilder/Katze} & \includegraphics[width=0.49\textwidth]{Bilder/Katze} \\
\end{tabular}


\the\textwidth

\the\linewidth

\the\columnwidth

\begin{itemize}
	\item \the\textwidth
	\item \the\linewidth
	\item \the\columnwidth
\end{itemize}

%!TeX root=Hauptdokument.tex
\chapter{Einführung}

\blindtext[200]

\blindtext[200]

\blindtext[200]


%!TeX root=Hauptdokument.tex
\chapter{Hauptteil}

\blindtext[200]

\blindtext[200]

\blindtext[200]

%!TeX root=Hauptdokument.tex
\chapter{Fazit}

\blindtext[20]

\blindtext[20]

\blindtext[20]


\end{document}