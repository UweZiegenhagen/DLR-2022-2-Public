%!TEX TS-program = Arara
% arara: pdflatex
% arara: nomencl
% arara: pdflatex
\documentclass[12pt,ngerman]{scrartcl}

\usepackage{babel}
\usepackage{siunitx}
\usepackage[nomentbl,german]{nomencl}
\makenomenclature

\begin{document}

\section*{Gleichungen}

\begin{equation}
E=mc^2
\end{equation}%

% \nomenclature[⟨prefix⟩]{⟨symbol ⟩}{⟨description⟩}{⟨units⟩}{⟨note⟩} 
\nomenclature{$E$}{Energie in Watt}{\meter/\second\squared}{Notiz}%

\(
E =  100m % Ganz falsch, denn Einheiten setzt man aufrecht
\)

\(
E =  100\si{\meter} % Besser, aber nicht gut, der Abstand zwischen 100 und m fehlt
\)


\(
E =  \SI{100}{\meter}  % this is the way!
\)

\printnomenclature

\end{document}