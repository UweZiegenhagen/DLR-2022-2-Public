%!TEX TS-program = Arara
% arara: pdflatex: {shell: yes}
% arara: biber
% arara: pdflatex: {shell: yes}

\documentclass[12pt,ngerman,parskip=half]{scrartcl}

\usepackage[utf8]{inputenc}
\usepackage[T1]{fontenc}
\usepackage{booktabs}
\usepackage{babel}
\usepackage{graphicx}
\usepackage{paralist}
\usepackage{xcolor}
\usepackage{blindtext}

\usepackage[style=authortitle-icomp,backend=biber]{biblatex}
\usepackage[babel,german=quotes]{csquotes}

\addbibresource{Literaturverzeichnis.bib}

\begin{document}

\blindtext 

\cite{Knuth:1984}

\blindtext 
 
\cite{ziegenhagen:2017}

\parencite{Knuth:1984}

\parencite{ziegenhagen:2017}

\blindtext\footcite{Knuth:1984}

\citeauthor{ziegenhagen:2017} hat in seinem im Jahr \citeyear{ziegenhagen:2017} erschienenen Artikel \citetitle{ziegenhagen:2017} die Einrichtung eines \TeX-Servers beschrieben.
  

\printbibliography

\end{document}