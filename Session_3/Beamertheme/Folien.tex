% !TEX TS-program = LuaLaTeX
\documentclass[serif,ngerman]{beamer}

\usepackage[T1]{fontenc}
\usepackage{booktabs}
\usepackage{graphicx}
\usepackage[babel,german=quotes]{csquotes}
\usepackage{listings}
\usepackage{amsmath}
\usepackage{lastpage}
\usepackage{rotating}
\usepackage{babel}
\usepackage{tikz,textcomp}

\usepackage{xcolor}
\definecolor{chamois}{RGB}{255,255,240}
\definecolor{darkbrown}{RGB}{124,79,0}
\definecolor{UniBlue}{RGB}{83,101,130}

\definecolor{hellgelb}{rgb}{1,1,0.8}
\definecolor{colKeys}{rgb}{0,0,1}
\definecolor{colIdentifier}{rgb}{0,0,0}
\definecolor{colComments}{rgb}{1,0,0}
\definecolor{colString}{rgb}{0,0.5,0}

\usepackage{xkeyval}
\usepackage{fontspec}
 
\setmainfont{Frutiger Next Pro}

\usepackage{graphicx}
\usepackage{csquotes}
 
\lstset{%
float=hbp,%
basicstyle=\ttfamily\small, %
identifierstyle=\color{colIdentifier}, %
keywordstyle=\color{darkbrown}, %
stringstyle=\color{colString}, %
commentstyle=\color{UniBlue}, %
columns=flexible, %
tabsize=2, %
frame=single, %
extendedchars=true, %
showspaces=false, %
showstringspaces=false, %
numbers=left, %
numberstyle=\tiny, %
breaklines=true, %
backgroundcolor=\color{hellgelb}, %
breakautoindent=true, %
captionpos=b%
}

\usepackage{progressbar}

\progressbarchange{width=0.85\textwidth, heighta=0.1cm, filledcolor=
brown, emptycolor=chamois, linecolor= chamois,borderwidth=1pt, tickswidth
=1pt, roundnessr=0.1, subdivisions=20,tickscolor=brown}

%\usepackage{bera}
%\usepackage[default]{sourcecodepro}

\usepackage{tikz}
\newcommand{\topline}{%
  \tikz[remember picture,overlay] {%
    \draw[brown,ultra thick] ([yshift=-1cm]current page.north west)-- ([yshift=-1cm,xshift=\paperwidth]current page.north west);}}


\addtobeamertemplate{frametitle}{}{\topline%
}


% http://tex.stackexchange.com/questions/7801/how-to-pass-in-a-ratio-in-fraction-form-as-opposed-to-decimal-form-to-table-co/7806#7806
\makeatletter
\newcommand\ratio[2]{%
  \strip@pt\dimexpr#1pt/#2\relax
}
\makeatother

%\setbeamertemplate{footline}{%
%\hspace*{1cm}\progressbar{\ratio{\thepage}{\getpagerefnumber{LastPage}}}\hfill\textcolor{brown}{\footnotesize\thepage\ | \pageref{LastPage}}\hspace{1em}\vspace*{1em}}

\setbeamertemplate{footline}{%
\hspace*{1cm}\progressbar{\ratio{\thepage}{\getpagerefnumber{LastPage}}}\vspace*{3em}}


\setbeamertemplate{navigation symbols}{}
\setbeamercolor{background canvas}{bg=chamois}
\setbeamercolor{itemize item}{fg=brown}
\setbeamertemplate{itemize item}{\maltese}
\setbeamercolor{itemize subitem}{fg=brown}
\setbeamertemplate{itemize subitem}{$\infty$}


%\textasteriskcentered

\setbeamercolor{title}{fg=UniBlue}
\setbeamercolor{frametitle}{fg=UniBlue}
\setbeamercolor{structure}{fg=UniBlue}

\setbeamercolor{author}{fg=darkbrown}
\setbeamercolor{date}{fg=darkbrown}

\setbeamercolor{block title}{bg=darkbrown!40,fg=darkbrown!90}
\setbeamercolor{block body}{bg=darkbrown!20,fg=UniBlue}
\addtobeamertemplate{block begin}{%
  \setlength{\textwidth}{0.8\textwidth}%
}{}


\setbeamercolor{block title alerted}{bg=yellow!60,fg=red}
\setbeamercolor{block body alerted}{bg=hellgelb!80,fg=UniBlue}


\author{Max Mustermann}
\title{Titel der Präsentation}
\subtitle{Untertitel der Präsentation}


\begin{document}

\frame{
\maketitle
}

\frame{
\frametitle{Inhalt}

\tableofcontents

}

\section{Einleitung}

\subsection{Mahatma Gandhi}

\frame{
\frametitle{\textsc{Mahatma Gandhi} \footnotesize{(2. Oktober 1869 in Porbandar, Gujarat)}}

\begin{itemize}
\item  indischer Rechtsanwalt \textbf{dasfsdfsd} \textbf{\textit{fsdfs}}
\item  Widerstandskämpfer
\item  Revolutionär
\item Publizist und Morallehrer
\item Asket und Pazifist
\begin{itemize}
\item 
\item 
\item 
\end{itemize}
\end{itemize}
}

\frame{
\frametitle{Titel der Folie 1}

\begin{itemize}
\item 
\item 
\item 
\end{itemize}
}

\frame{
\frametitle{Titel der Folie 2}

\begin{itemize}
\item 
\item 
\item 
\end{itemize}
}


\frame{
\frametitle{Titel der Folie 3}

\begin{itemize}
\item 
\item 
\item 
\end{itemize}
}


\frame{
\frametitle{Titel der Folie 4}

\begin{itemize}
\item 
\item 
\item 
\end{itemize}
}


\frame{
\frametitle{Titel der Folie 5}

\begin{itemize}
\item 
\item 
\item 
\end{itemize}
}


\frame{
\frametitle{Titel der Folie 6}

\begin{itemize}
\item 
\item 
\item 
\end{itemize}
}


\frame{
\frametitle{Titel der Folie 7}

\begin{itemize}
\item 
\item 
\item 
\end{itemize}
}


\frame{
\frametitle{Titel der Folie 8}

\begin{itemize}
\item 
\item 
\item 
\end{itemize}
}

\frame{
\frametitle{Titel der Folie 9}

\begin{itemize}
\item 
\item 
\item 
\end{itemize}
}



\end{document}